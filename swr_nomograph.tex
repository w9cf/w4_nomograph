\documentclass[12pt]{article}
\usepackage[margin=25.4mm]{geometry}
\usepackage{amsmath}
\usepackage{amssymb}
\usepackage{graphicx}
\usepackage{tikz}
\usetikzlibrary{automata, positioning, arrows}
\begin{document}
\let\xxxhat\hat
\let\xxxvec\vec
\renewcommand{\hat}[1]{{\boldsymbol {\xxxhat {#1}} }}
\renewcommand{\vec}[1]{\boldsymbol {#1}}
\newcommand{\dbar}{{ d\hskip -.25em \raisebox{.3em}{-}  }}

\title{SWR nomograph}
\author{Kevin E. Schmidt, W9CF\\
6510 South Roosevelt Street\\
Tempe, AZ 85283\\
}
\maketitle

The voltage standing wave ratio is given by the ratio of the
maximum to minimum peak voltage along a transmission line. For an ideal
line, the maximum occurs when the forward and reverse waves are in
phase and the minimum when they are out of phase. So if their magnitudes
are $V_f$ and $V_r$, and assuming $V_r$ represents a reflected wave
so it is smaller than $V_f$,
\begin{equation}
{\rm SWR} = s = \frac{V_f+V_r}{V_f-V_r} \,.
\end{equation}

The Drake W-4 measures
the forward power $P_f = \frac{V_f^2}{2Z_0}$
and the reverse power $P_r = \frac{V_r^2}{2Z_0}$, where
the characteristic impedance, $Z_0$, is 50 ohms.
Substituting we see
\begin{equation}
\label{eq.swr}
{\rm SWR} =
\frac{1+\sqrt{\frac{P_r}{P_f}}}
{1-\sqrt{\frac{P_r}{P_f}}} \,.
\end{equation}
This is the equation that the Drake nomograph solves.

Since the quantity we want is a function of the ratio of the powers,
and the nomograph will give a linear combination of the left and right
axes, we need these axes to be logarithmic so that taking a linear
combination of the logarithms corresponds to taking the ratio we desire.

Following Drake, we place the axes vertically, and
take the left axis to be forward power, the
right axis to be reflected power, and place the swr axis in between them.
In that case, since we need to divide the powers, the axis need to run
in the opposite directions so that the interpolation subtracts the logs,
so we take the forward power increasing from the bottom to the top,
and the reverse increasing from the top to the bottom.

I take the power axes to be equal height $H$, on the paper, with
$P_f^{~\rm max} \leq P_f \leq P_f^{~\rm min}$
and $P_r^{~\rm max} \leq P_r \leq P_r^{~\rm min}$.

Placing $P_f^{\rm min}$ at the bottom of the forward axis, and
$P_r^{\rm min}$ at the top of the reverse axis, the distance from the
bottom of the axes is given by
\begin{eqnarray}
d_f &=& H \frac{\ln\left (\frac{P_f}{P_f^{~\rm min}}\right)}
{\ln\left (\frac{~P_f^{\rm max}}{P_f^{~\rm min}}\right)}
\nonumber\\
d_r &=& -H \frac{\ln\left (\frac{P_r}{P_r^{~\rm max}}\right)}
{\ln\left (\frac{~P_r^{\rm max}}{P_r^{~\rm min}}\right)} \,.
\end{eqnarray}
Taking the horizontal distance between these axes to be the width $W$
on the paper, we place the forward axis at $0$, the reverse axis at
$W$, and the swr axis at $\alpha W$ where $0 < \alpha < 1$ and we
must
choose $\alpha$ to get the distance along this axis to be a monotonic
function of $\frac{P_r}{P_f}$. The distance along the SWR axis is then
\begin{equation}
d_s = (1-\alpha) d_f + \alpha d_r\,,
\end{equation}
and to get the logs to have the same factor, we require
\begin{equation}
\frac{1-\alpha}
{\ln\left (\frac{~P_f^{\rm max}}{P_f^{~\rm min}}\right)}
=
\frac{\alpha}
{\ln\left (\frac{~P_r^{\rm max}}{P_r^{~\rm min}}\right)} \,,
\end{equation}
with solution
\begin{equation}
\alpha =
\frac{\ln\left (\frac{~P_r^{\rm max}}{P_r^{~\rm min}}\right)}
{\ln\left (\frac{~P_r^{\rm max}}{P_r^{~\rm min}}\right)
+
\ln\left (\frac{~P_f^{\rm max}}{P_f^{~\rm min}}\right)} \,.
\end{equation}

Substituting back, we find
\begin{equation}
d_s = H\frac{
\ln \left( \frac{P_f}{P_r} \right)
+ \ln\left ( \frac{P_r^{~\rm max}}{P_f^{~\rm min}} \right)
}
{\ln\left (\frac{~P_r^{\rm max}}{P_r^{~\rm min}}\right)
+
\ln\left (\frac{~P_f^{\rm max}}{P_f^{~\rm min}}\right)} \,.
\end{equation}
Solving Eq. \ref{eq.swr} for $s$ the SWR and substituting
\begin{equation}
d_s = H\frac{
\ln \left [\left( \frac{s+1}{s-1} \right)^2 \right ]
+ \ln\left ( \frac{P_r^{~\rm max}}{P_f^{~\rm min}} \right)
}
{\ln\left (\frac{~P_r^{\rm max}}{P_r^{~\rm min}}\right)
+
\ln\left (\frac{~P_f^{\rm max}}{P_f^{~\rm min}}\right)} \,,
\end{equation}
gives the position where a particular label will be placed on the
SWR axis.
Notice that since all factors contain a log, the base of the log
cancels, and any base can be used.

\end{document}
